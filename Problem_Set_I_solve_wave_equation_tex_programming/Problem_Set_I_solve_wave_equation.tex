\documentclass[10pt,fleqn,reqno,a4paper]{article}
\usepackage[utf8]{inputenc}
\usepackage{amsfonts}
\usepackage{amssymb }
%\usepackage{ngerman}
\usepackage[german, english]{babel}
%\usepackage[english]{babel}
\usepackage{ upgreek }
%\usepackage{biblatex}
\usepackage[square,sort,comma,numbers]{natbib}% bibliography style for support for urls in the 
\bibliographystyle{unsrtnat}%abbrvnat
%bibliography
%\usepackage{noweb}
%\pagestyle{plain}% plain without headlines, with something more \pagestyle{noweb}
%\noweboptions{shift,smallcode,longchunks}%german,smallcode,longchunks

\usepackage{graphicx}
\usepackage{caption}
\usepackage{subcaption}
\usepackage{graphviz}
\usepackage[hidelinks]{hyperref}
\usepackage[dvipsnames]{xcolor}
\usepackage{comment}
\usepackage{geometry}
\usepackage{tikz}
\geometry{a4paper, portrait,left=2.5cm, right=2.5cm, top=2cm, bottom=2cm}
\usepackage[T1]{fontenc}
\usepackage{lmodern}
\usepackage[tbtags,fleqn, % Platzierung der Formel-Tags;% es gibt auch centertags
sumlimits,
% Platzierung der Summationsgrenzen
% (oberhalb/unterhalb)
intlimits,
% Platzierung der Integrationsgrenzen
% (oberhalb/unterhalb)
namelimits]
% Platzierung der Grenzen
% (oberhalb/unterhalb) bei Funktionen
{amsmath}
%\setlength{\mathindent}{0pt}

%    \usepackage{mathpazo}
%\usepackage[mathpazo]{flexisym}
%\usepackage{breqn}% automatic line break in equations
\usepackage{icomma}% für die richtige Kommadarstellung in Formeln und in Texten
\usepackage[toc,page]{appendix}
\usepackage{listings}% TXT Dateien in LaTeX einbinden
\usepackage{ mathrsfs }
\usepackage{ mathtools}
%\usepackage{dsfont}
% Notation page

%\usepackage[nomain]{glossaries}

\usepackage[toc,page]{appendix}

\usepackage{minted}
\definecolor{lightgraycolor}{rgb}{0.95,0.95,0.95}
\usemintedstyle{manni}
% Farbige Formeln richtig einfärben über die group Umgebung
\def\mathcolor#1#{\@mathcolor{#1}}
\def\@mathcolor#1#2#3{%
	\protect\leavevmode
	\begingroup\color#1{#2}#3\endgroup
}
\newcommand*{\opcolor}[2]{\mathop{\color{#1}{#2}}}
%%%%%%%%%%%%%% important variables
\newcommand{\nphi}{{\color{cyan!70!black}\phi}}
\newcommand{\neta}{{\color{cyan!70!black}\eta}}
\newcommand{\nchi}{{\color{cyan!70!black}\chi}}

\newcommand{\nL}{{\color{cyan!70!black}L}}
\newcommand{\nnL}{{\color{cyan!70!black}\mathcal{L} } }
\newcommand{\ntL}{\tilde{{\color{cyan!70!black}L}}}
\newcommand{\nhL}{\hat{{\color{cyan!70!black}L}}}
\newcommand{\nnnL}{{\color{cyan!70!black}\mathscr{L}}}
\newcommand{\nH}{{\color{cyan!70!black}H}}
\newcommand{\nnH}{{\color{cyan!70!black}\mathcal{H}}}
%\newcommand{\nD}{{\color{green!50!black}D}
\newcommand{\nz}{{\color{brown!80!black}z}}
%\newcommand{\mA}{{\color{green!50!black}A}}
%\newcommand{\nC}{{\color{cyan!70!black}C}}

%---
\newcommand{\nR}{{\color{cyan!70!black} R } }

%%%%%%%%%%%%%% generalized coordinates

%\newcommand{\na}{{\color{blue!50!black}a}
\newcommand{\nx}{{\color{gray}x}}
%\newcommand{\nX}{{\color{gray}X}
\newcommand{\nnx}{{\color{green!50!black}x}}
%\newcommand{\nx}{{\color{cyan!70!black}x}
\newcommand{\ny}{{\color{gray}y}}
%\newcommand{\nz}{{\color{green!30!black}\textit{z}}
%\newcommand{\nA}{{\color{green!50!black}A}
\newcommand{\nw}{{\color{gray}w}}
%\newcommand{\nW}{{\color{gray}W}
\newcommand{\nt}{{\color{gray}t}}
\newcommand{\ntau}{{\color{gray}\tau}}
%\newcommand{\npi}{{\color{gray}\pi}
\newcommand{\nq}{{\color{gray}q}}%!90!black
\newcommand{\cq}{{\color{gray}q}}
\newcommand{\dcq}{\dot{\cq}}
\newcommand{\ddcq}{\ddot{\cq}}

\newcommand{\dddcq}{\dddot{\cq}}
\newcommand{\ddddcq}{\ddot{\ddot{\cq}} }

\newcommand{\cQ}{{\color{gray}Q}}%!90!black
\newcommand{\nQ}{{\color{gray}Q}}
\newcommand{\dcQ}{\dot{\cQ}}
\newcommand{\np}{{\color{gray}p}}%!60!black
\newcommand{\cp}{{\color{gray}p}}%!60!black
\newcommand{\dcp}{{\dot{\cp} }}%!60!black
\newcommand{\cpi}{{\color{gray}\pi}}
\newcommand{\cphi}{{\color{gray}\varphi}}
\newcommand{\cA}{{\color{gray}A }  }

%---

\newcommand{\ct}{{\color{gray}t}}
\newcommand{\ccr}{{\color{gray}r}}
\newcommand{\ctheta}{{\color{gray}\theta}}
\newcommand{\cx}{{\color{gray}x}}
%\newcommand{\fa}{{\color{gray} a } }

\newcommand{\fN}{{\color{gray} N } }
%\newcommand{\fA}{{\color{gray}A}}
\newcommand{\fvA}{{\color{gray}\mathcal{A}}}


%%%%%%%%%%%%%% functions

\newcommand{\nW}{{\color{blue!70!black}W}}
\newcommand{\feta}{{\color{blue!70!black}\eta}}
\newcommand{\fu}{{\color{blue!70!black}u}}
\newcommand{\dnnu}{\dot{\nnu}}
\newcommand{\nnv}{{\color{blue!70!black}v}}
\newcommand{\nsin}{{\color{blue!70!black}\sin}}
\newcommand{\ncos}{{\color{blue!70!black}\cos}}
\newcommand{\ntan}{{\color{blue!70!black}\tan}}
\newcommand{\narctan}{{\color{blue!70!black}\arctan}}
\newcommand{\ncosh}{{\color{blue!70!black}\cosh}}
\newcommand{\nsinh}{{\color{blue!70!black}\sinh}}
\newcommand{\nexp}{{\color{blue!70!black}\exp}}
\newcommand{\nf}{{\color{blue!70!black}f}}
\newcommand{\nh}{{\color{blue!70!black}h}}
\newcommand{\nA}{{\color{blue!70!black}A}}
\newcommand{\nB}{{\color{blue!70!black}B}}
\newcommand{\nC}{{\color{blue!70!black}C}}
\newcommand{\nV}{{\color{blue!70!black}V}}
\newcommand{\nxi}{{\color{blue!70!black}\xi}}
\newcommand{\dif}{\mathrm{{\color{blue!70!black}d } } }
\newcommand{\ndelta}{{\color{blue!70!black}\delta}}
\newcommand{\nN}{{\color{blue!70!black}N}}
\newcommand{\fphi}{{\color{blue!70!black}\phi}}
\newcommand{\nPhi}{{\color{blue!70!black}\Phi}}
\newcommand{\nnmu}{{\color{blue!70!black}\mu}}
\newcommand{\dnnmu}{\dot{\nnmu}}
\newcommand{\nr}{{\color{blue!70!black}r}}
\newcommand{\nT}{{\color{blue!70!black}T}}
\newcommand{\nln}{{\color{blue!70!black}\ln}}
\newcommand{\nE}{{\color{blue!70!black}E}}
\newcommand{\nK}{{\color{blue!70!black}K}}
\newcommand{\nng}{{\color{blue!70!black}g}}
\newcommand{\nnq}{{\color{blue!70!black}q}}
\newcommand{\npi}{{\color{blue!70!black}\varpi}}
\newcommand{\nlambda}{{\color{blue!70!black}\lambda}}
\newcommand{\nZ}{{\color{blue!70!black}Z}}
\newcommand{\nX}{{\color{blue!70!black}X}}
\newcommand{\nY}{{\color{blue!70!black}Y}}
\newcommand{\nF}{{\color{blue!70!black}F}}

%--- vector
\newcommand{\vfu}{{\vec{\fu}}}
\newcommand{\vfS}{{\vec{\fS}}}


%--- matrix
\newcommand{\MA}{{\mathbf{\fA}}}




%---
%\newcommand{\fa}{{\color{blue!70!black} a } }
\newcommand{\fg}{{\color{blue!70!black} g } }
%\newcommand{\fN}{{\color{blue!70!black} N } }
\newcommand{\fA}{{\color{blue!70!black}A}}
%\newcommand{\fvA}{{\color{blue!70!black}\mathcal{A}}}

\newcommand{\fK}{{\color{blue!70!black} K } }
\newcommand{\fE}{{\color{blue!70!black} E } }
\newcommand{\fW}{{\color{blue!70!black} W } }

\newcommand{\fc}{{\color{blue!70!black} c } }

\newcommand{\fsin}{{\color{blue!70!black}\sin}}
\newcommand{\fcos}{{\color{blue!70!black}\cos}}
\newcommand{\ftan}{{\color{blue!70!black}\tan}}
\newcommand{\farctan}{{\color{blue!70!black}\arctan}}
\newcommand{\fcosh}{{\color{blue!70!black}\cosh}}
\newcommand{\fsinh}{{\color{blue!70!black}\sinh}}
\newcommand{\fexp}{{\color{blue!70!black}\exp}}



\newcommand{\fl}{{\color{blue!70!black}l}}
\newcommand{\fS}{{\color{blue!70!black}S}}
\newcommand{\fGamma}{{\color{blue!70!black}\Gamma}}
\newcommand{\fmu}{{\color{blue!70!black}\mu}}
\newcommand{\fb}{{\color{blue!70!black}b}}

\newcommand{\fG}{{\color{blue!70!black}G}}
\newcommand{\fvG}{{\color{blue!70!black}\mathcal{G}}}
\newcommand{\fs}{{\color{blue!70!black}s}}
\newcommand{\fT}{{\color{blue!70!black}T}}
\newcommand{\fB}{{\color{blue!70!black}B}}
\newcommand{\fvv}{\vec{\color{blue!70!black}v}}
\newcommand{\ff}{{\color{blue!70!black}f}}
\newcommand{\detg}{\sqrt{-{\color{blue!70!black}\fg} } }
\newcommand{\fC}{{\color{blue!70!black}C}}
\newcommand{\fepsilon}{{\color{blue!70!black}\varepsilon}}
\newcommand{\fnu}{{\color{blue!70!black}\nu}}
\newcommand{\fF}{{\color{blue!70!black}F}}
\newcommand{\fchi}{{\color{blue!70!black}\chi}}

\newcommand{\finfty}{{\color{blue!70!black}\infty}}

%%%%%%%%%%%%%% Operators

\newcommand{\oupdelta}{{\color{blue!70!black}\updelta}}
\newcommand{\ovary}{{\color{blue!70!black}\updelta}}
\newcommand{\osum}{{\opcolor{blue!70!black}{\sum}}}
\newcommand{\ooint}{{\opcolor{blue!70!black}{\sum}}}
\newcommand{\opartial}{{\color{blue!70!black}\partial}}
%%%%%%%%%%%%%% indecies

\newcommand{\nni}{{\color{red!70!black}i}}
\newcommand{\nnj}{{\color{red!70!black}j}}
\newcommand{\nk}{{\color{red!70!black}k}}
\newcommand{\nalpha}{{\color{red!70!black}\alpha}}
\newcommand{\nbeta}{{\color{red!70!black}\beta}}
\newcommand{\ngamma}{{\color{red!70!black}\gamma}}
\newcommand{\nm}{{\color{red!70!black}m}}
\newcommand{\nmu}{{\color{red!70!black}\mu}}
\newcommand{\nmub[1]}{\bar{\nmu}_{#1}}

\newcommand{\nn}{{\color{red!70!black}n}}
\newcommand{\nl}{{\color{red!70!black}l}}


\newcommand{\na}{{\color{red!70!black}a}}
\newcommand{\nta}{{\color{red!70!black}\tilde{a} } }
\newcommand{\nb}{{\color{red!70!black}b } }
\newcommand{\ntb}{{\color{red!70!black}\tilde{b}} }

%---
\newcommand{\iit}{{\color{red!70!black}t}}
\newcommand{\ir}{{\color{red!70!black}r}}
\newcommand{\itheta}{{\color{red!70!black}\theta}}
\newcommand{\iphi}{{\color{red!70!black}\varphi}}
\newcommand{\ia}{{\color{red!70!black}a}}
\newcommand{\ib}{{\color{red!70!black}b}}
\newcommand{\ic}{{\color{red!70!black}c}}
\newcommand{\id}{{\color{red!70!black}d}}
\newcommand{\ie}{{\color{red!70!black}e}}
\newcommand{\iif}{{\color{red!70!black}f}}
\newcommand{\ii}{{\color{red!70!black}i}}
\newcommand{\iij}{{\color{red!70!black}j}}
\newcommand{\iin}{{\color{red!70!black}n}}
\newcommand{\iim}{{\color{red!70!black}m}}
\newcommand{\il}{{\color{red!70!black}l}}
\newcommand{\ik}{{\color{red!70!black}k}}
\newcommand{\ialpha}{{\color{red!70!black}\alpha}}



\newcommand{\miin}{{\color{red!70!black}\boldsymbol{n}}}
\newcommand{\mim}{{\color{red!70!black}\boldsymbol{m}}}


%%%%%%%%%%%%%% derivatives


\newcommand{\npartial}{{\color{blue!70!black}\partial}}
\newcommand{\tpartial}{{\color{blue!70!black}\partial_{\ct}}}
\newcommand{\dcphi}{\dot{\cphi}}
\newcommand{\ddcphi}{\ddot{\cphi}}
\newcommand{\dcA}{\dot{\cA}}
\newcommand{\ddcA}{\ddot{\cA}}
\newcommand{\dnq}{\dot{\nnq}}

\newcommand{\dcphii}{\dot{\cphi}_{\nni}}
\newcommand{\ddcphii}{\ddot{\cphi}_{\nni}}
\newcommand{\dcAi}{\dot{\cA}_{\nni}}
\newcommand{\ddcAi}{\ddot{\cA}_{\nni}}
\newcommand{\dnqi}{\dot{\nnq}_{\nni}}
%\newcommand{\partie[1]}{ \npartial_{\bar{\nmu}_{\scriptscriptstyle{#1}}}}
\newcommand{\partie[1]}{ \npartial_{\bar{\nmu}_{#1}}}

\newcommand{\da}{\dot{\fa}}
\newcommand{\dda}{\ddot{\fa}}
\newcommand{\ddda}{\fa^{(3)}}
\newcommand{\dddda}{\fa^{(4)}}
\newcommand{\dN}{\dot{\fN}}
\newcommand{\ddN}{\ddot{\fN}}
\newcommand{\dddN}{\fN^{(3)}}
\newcommand{\dA}{\dot{\fA_{\phantom{,}}}\!}
%\newcommand{\ddA}{\color{blue!70!black}\ddot{A}}
\newcommand{\ddA}{{\ddot{\fA_{\phantom{,}}}}\! }%\color{gray}
\newcommand{\dmu}{\dot{\fmu}}
\newcommand{\du}{\dot{\fu}}
\newcommand{\ddu}{\ddot{\fu}}
\newcommand{\dddu}{\fu^{(3)}}
\newcommand{\dphi}{\dot{\fphi}}
\newcommand{\dvA}{\dot{\fvA_{\phantom{a}}}\!\!}
\newcommand{\ddvA}{\ddot{\fvA_{\phantom{a}}}\!\!}
\newcommand{\depsilon}{\dot{\fepsilon}}

%%%%%%%%%%%%%% special variants

\newcommand{\vv}{v}
\newcommand{\MM}{M}
\newcommand{\va}{{\color{red!70!black}\mathrm{a}}}
\newcommand{\vV}{{\color{violet!80!black}V}}

\definecolor{lightgraycolor}{rgb}{0.95,0.95,0.95}
\begin{document}%\selectlanguage{english}
\section{Problem Set I solving wave equation}

\begin{align}
	\frac{\opartial^2\nphi}{\opartial^2\ct} = c^2\frac{\opartial^2\nphi}{\opartial^2\cx}
\end{align}

\subsection{fully first order formulation}

\begin{align}
	\neta = \nphi_{,\ct}, \quad \nchi = \nphi_{,\cx}
\end{align}



$ \neta(\ct,\cx) \nchi(\ct,\cx) \vfu(\nphi,\neta,\nchi) $


\begin{align}
	\vfu_{,\ct}+\MA\vfu_{,\cx}=\vfS
\end{align}



\subsection{initial condition}

\begin{align}
	\nphi(0, \cx)=e^{\fsin^2\left(\frac{\pi \cx}{L}\right)} -1, \quad 0 \leq \cx \leq L
\end{align}
with periodic condition:
\begin{align}
	\nphi(\ct,\cx) = \nphi(\ct, \cx \pm L)
\end{align}


\section{Program}
\begin{minted}[linenos=true,bgcolor=lightgraycolor,numberblanklines=true,showspaces=false,breaklines=true]{cpp}
#include <cstdio>
#include <cmath>
#include <fstream>
#include <iostream>
using namespace std ;

void output(int ti, int xi, double t, double x[], double phi[][2]);
void init(double t, double x[], double phi[][2], double eta[][2], double chi[][2], int xSteps, double dx, double L);
void boundaryCondition(int ti, int xSteps, double phi[][2], double eta[][2], double chi[][2]);
double secondOrderSpatial(double funct2[][2], int xi, double dx);
void forwardEulerMethod(double funct[][2], double funct2[][2], double dt, int xi, double dx, double factor, int deriv);
void solvingWaveEquation(double phi[][2], double eta[][2], double chi[][2], double t, double dt, double x[], double dx, double CSpeed, int xSteps, int tSteps);
void updateFunc(int xSteps, double phi[][2], double eta[][2], double chi[][2]);
void gnuplot();

void output(int ti, int xi, double t, double x[], double phi[][2]){
    // x phi
    cout << x[xi] << ' ' <<  phi[xi][ti] << endl;
};

void init(double t, double x[], double phi[][2], double eta[][2], double chi[][2], int xSteps, double dx, double L){
    cout << "reset" << endl;
    cout << "set xrange [0:1]" << endl;
    cout << "set yrange [-10:10]" << endl;
    gnuplot();
    for (int i = 2; i < xSteps-2; i=i+1) {
        phi[i][0] = exp(pow(sin(M_PI/L*((i-2)*dx)),2))-1;
        chi[i][0] = phi[i][0];
        //chi[i][0] = exp(pow(sin(M_PI/L*((i-2)*dx)),2))*2*sin(M_PI/L*((i-2)*dx))*cos(M_PI/L*((i-2)*dx))*M_PI/L;
        //chi[i][0] = (i-2)*dx;
        //chi[i][0] = sin(M_PI/L*((i-2)*dx));
        //chi[i][0] = 1;
        //eta[i][0] = chi[i][0];
        //eta[i][0] = 0;
        eta[i][0] = 1;
        //eta[i][0] = pow(sin(M_PI/L*((i-2)*dx)),2);
        //eta[i][0] = exp(pow(sin(M_PI/L*((i-2)*dx)),2))*2*sin(M_PI/L*((i-2)*dx))*cos(M_PI/L*((i-2)*dx))*M_PI/L;
        x[i]=(i-2)*dx;
        output(0, i, t, x, phi);
	}
	x[xSteps-2]=(xSteps-4)*dx;
    boundaryCondition(0, xSteps, phi, eta, chi);
    output(0, (xSteps-2), t, x, phi);
};

void boundaryCondition(int ti, int xSteps, double phi[][2], double eta[][2], double chi[][2]){
    phi[0][ti] = phi[xSteps-4][ti];
    eta[0][ti] = eta[xSteps-4][ti];
    chi[0][ti] = chi[xSteps-4][ti];
    phi[1][ti] = phi[xSteps-3][ti];
    eta[1][ti] = eta[xSteps-3][ti];
    chi[1][ti] = chi[xSteps-3][ti];
    phi[xSteps-2][ti] = phi[2][ti];
    eta[xSteps-2][ti] = eta[2][ti];
    chi[xSteps-2][ti] = chi[2][ti];
    phi[xSteps-1][ti] = phi[3][ti];
    eta[xSteps-1][ti] = eta[3][ti];
    chi[xSteps-1][ti] = chi[3][ti];
    phi[xSteps][ti] = phi[4][ti];
    eta[xSteps][ti] = eta[4][ti];
    chi[xSteps][ti] = chi[4][ti];
};

double secondOrderSpatial(double funct2[][2], int xi, double dx){
    return (funct2[xi+1][0]-funct2[xi-1][0])/(2*dx);
};

void forwardEulerMethod(double funct[][2], double funct2[][2], double dt, int xi, double dx, double factor, int deriv){
    if (deriv == 0) {
        funct[xi][1]=funct[xi][0]+factor*dt*funct2[xi][0];
    } else {
        funct[xi][1]=funct[xi][0]+factor*dt*secondOrderSpatial(funct2, xi, dx);
    }
};

void solvingWaveEquation(double phi[][2], double eta[][2], double chi[][2], double t, double dt, double x[], double dx, double CSpeed, int xSteps, int tSteps){
    for (int j = 1; j < tSteps; j=j+1) {
        t=j*dt;
        gnuplot();
        for (int i = 2; i < xSteps-2; i=i+1) {
            forwardEulerMethod(phi, eta, dt, i, dx, 1, 0);
            forwardEulerMethod(eta, chi, dt, i, dx, pow(CSpeed, 2), 1);
            forwardEulerMethod(chi, eta, dt, i, dx, 1, 1);
            output(1, i, t, x, phi);
        };
        boundaryCondition(1, xSteps, phi, eta, chi);
        output(1, (xSteps-2), t, x, phi);
        cout << "elpased time" << endl;
        updateFunc(xSteps, phi, eta, chi);
    };
};

void updateFunc(int xSteps, double phi[][2], double eta[][2], double chi[][2]){
    for (int i = 0; i <= xSteps; i=i+1) {
        phi[i][0] = phi[i][1];
        chi[i][0] = chi[i][1];
        eta[i][0] = eta[i][1];
	}
};

void gnuplot(){
    cout << "plot '-' w l" << endl;
};

int main(int argc, char** argv)
{
    const double CSpeed = 1;
    const double CMax = 0.005;
    const double dx = stod(argv[1]); //
    const double L = 1; // gridSpace
    const double timeLength = 1;
    const double dt = CMax*dx/abs(CSpeed);
    const int nGhosts = 4;
    const int xSteps = int( L / dx ) + nGhosts;
    //const int tSteps = int (timeLength / dt );
    const int tSteps = int ( stod(argv[2]));

    double //
    x[xSteps],
    t=0,
    phi[xSteps][2],
    chi[xSteps][2],
    eta[xSteps][2]
    ;

    cout << "# parameters " << dx << ' ' << dt << ' ' <<  xSteps << endl;

    init(t, x, phi, eta, chi, xSteps, dx, L);

    // cases for solver
    //{{solving wave equation}}
    solvingWaveEquation(phi, eta, chi, t, dt, x, dx, CSpeed, xSteps, tSteps);

    //{{forth order spatial derivative}}
    //{{Runge Kutter solver}}
	return 0;
};
\end{minted}
\end{document}
